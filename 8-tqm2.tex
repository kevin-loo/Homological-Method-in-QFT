\section{Topological Quantum Mechanics-II}
\label{sec:tqm2}

Recall in \nameref{sec:tqm1}, we have discussed the first-order formalism of TQM such that in a local model with maps $\varphi: \Omega^\blt_{S^1} \to V\simeq \bR^{2n}$, the correlation map
\bea  \lan \cdots\ran_{free}: C_{-\blt}(\cW_{2n}) \to \widehat{\Omega}^{-\blt}_{2n}((\hbar ))\eea
intertwines $b$ with $\hbar\Delta$ and $B$ with $d_{2n}$.

In this section, we are going to glue this construction to a symplectic manifold and establish the algebraic index to universal Lie algebra
cohomology computations. The basic idea is to \emph{glue} the local model $\Sigma\to T^{Model}\subset X$. 
\textsc{Reference}: \cite{Gui:2019ldd}

\subsection{Gluing via Gelfand-Kazhdan formal geometry}
\begin{defn}
A \textbf{Harish-Chandra pair} is a pair $(\fg,K)$, where $\fg$ is a Lie algebra, $K$ is a Lie group, with 
\begin{itemize}
    \item an action of $K$ on $\fg$: $K\xrightarrow{\ \rho\ } \on{Aut}(\fg)$,
    \item a natural embedding: $\on{Lie}(K) \xhookrightarrow{\ i\ } \fg$, where $\on{Lie}(K)$ is the Lie algebra associated with $K$,
\end{itemize}
such that they are compatible:
\bea
\begin{tikzcd}
\on{Lie}(K) \ar[r, hook, "i"] \ar[dr, "d\rho"]
& \fg \ar[d,"adjoint"] \\
& \on{Der}(\fg)
\end{tikzcd}
\eea
\end{defn}

\begin{defn}
A \textbf{$(\fg,K)$-module} is a vector space $V$ with 
\begin{itemize}
    \item an action of $K$ on $V$: $K \xrightarrow{\ \varphi\ } \on{GL}(V)$,
    \item a Lie algebra morphism: $\fg\to \on{End}(V)$,
\end{itemize}
such that they are compatible:
\bea
\begin{tikzcd}
\on{Lie}(K) \ar[r, hook, "i"] \ar[dr, "d\varphi"]
& \fg \ar[d] \\
& \on{End}(V)
\end{tikzcd}
\eea
\end{defn}

\begin{defn}
A \textbf{flat $(\fg,K)$-bundle} over X is
\begin{itemize}
    \item a principal $K$-bundle $P \xrightarrow{\ \pi\ } X$,
    \item a $K$-equivariant $\fg$-valued 1-form $\gamma\in \Omega^1(P,\fg)$ on $P$,
\end{itemize}
satisfying the following conditions:
\bi[(1)]
    \item $\forall a\in \on{Lie}(K)$, let $\xi_a\in \on{Vect}(P)$ generated by $a$. Then we have the contraction $\gamma(\xi_a)=a$ such that
    \bea
    \begin{tikzcd}
        0 \ar[r] & \on{Lie}(K) \ar[r] \ar[dr, hook, "i"] & \on{Vect}(P) \ar[d, "\gamma"] \\
        & & \fg
    \end{tikzcd}
    \eea
    \item $\gamma$ satisfies the Maurer-Cartan equation
    \bea d\gamma +\hf \lsb \gamma,\gamma\rsb=0,\eea
    where $d$ is the de Rham differential on $P$, and $\lsb-,-\rsb$ is the Lie bracket in $\fg$.
\ei
\end{defn}

Given a flat $(\fg,K)$-bundle $P\to X$ and $(\fg,K)$-module $V$, let \bea\Omega^\blt(P,V)\coloneqq \Omega^\blt (P)\otimes V\eea
denote differential forms on $P$ valued in $V$.
It carries a connection
\bea \nabla^\gamma= d+\gamma :  \Omega^\blt(P,V)\to \Omega^{\blt+1}(P,V)\eea
which is \emph{flat} by the Maurer-Cartan equation.
The group $K$ acts on $\Omega^\blt(P)$ and $V$, and hence inducing a natural action on $\Omega^\blt(P,V)$.

Let 
\bea V_P\coloneqq P \times_K V\eea 
be the vector bundle on $X$ associated to the $K$-representation $V$. Let $\Omega^\blt(X;V_P)$ be differential forms
on $X$ valued in the bundle $V_P\to X$.
Similar to the usual principal bundle case, $\nabla^\gamma$ induces a flat connection on $V_P\to X$. This defines a (de Rham) chain complex $\lb \Omega^\blt(X;V_P), \nabla^\gamma\rb$, and $H^\blt(X;V_P)$ denotes the corresponding de Rham cohomology.

Next we discuss how to descend Lie algebra cohomologies to geometric objects on $X$.
\begin{defn}
Let $V$ be a $(\fg,K)$-module. Define the \textbf{$(\fg,K)$ relative Lie algebra cochain complex} 
$\lb C^\blt_{Lie}(\fg,K;V), \p_{Lie}\rb$
by
\bea C^p_{Lie}(\fg,K;V)=\on{Hom}_K \lb \asym^p\lb\fg/\on{Lie}(K)\rb,V\rb.\eea
\end{defn}
Here $\on{Hom}_K$ means $K$-equivariant linear maps. $\p_{Lie}$ is the Chevalley-Eilenberg differential if we view
$C^p_{Lie}(\fg,K;V)$ as a subspace of the Lie algebra cochain $C^p_{Lie}(\fg;V)$. Explicitly,
for $\alpha\in C^p_{Lie}(\fg,K;V)$,
\bea \lb \p_{Lie}\alpha\rb \lb a_1 \wedge \cdots \wedge a_{p+1}\rb
&=\sum_{i=1}^{p+1}(-1)^{i-1} a_i\cdot \alpha\lb a_1 \wedge \cdots \wedge \widehat{a}_i \wedge \cdots \wedge a_{p+1}\rb\\
&\quad +\sum_{i<j} (-1)^{i+j} \alpha\lb \lsb a_i,a_j\rsb \wedge \cdots \wedge \widehat{a}_i \wedge \cdots \wedge \widehat{a}_j \wedge \cdots \wedge a_{p+1}\rb.\eea
The corresponding cohomology is $H^\blt_{Lie}(\fg,K;V)$. 

Given a $(\fg,K)$-module $V$ and flat $(\fg,K)$-bundle $P\to X$ with the flat connection $\gamma\in\Omega^1(P,\fg)$. 
We can define the \textbf{descent map} from the $(\fg, K)$ relative Lie algebra cochain complex to
$V$-valued de Rham complex on $P$ by
\bea \on{desc}: \lb C^\blt_{Lie}(\fg,K;V),\p_{Lie}\rb\to \lb \Omega^\blt(X;V_P),\nabla^\gamma\rb, \quad \alpha\mapsto \alpha(\gamma,\cdots,\gamma)\eea
inducing the cohomology descent map
\bea \on{desc}: H^\blt_{Lie}(\fg,K;V)\to H^\blt (X;V_P).\eea

\subsection{Fedosov connection revisited}
Recall the (formal) Weyl algebras
\bea \cW_{2n}=\bR[[p_i,q^i]] ((\hbar)), \quad 
\cW_{2n}^+=\bR[[p_i,q^i]] [[\hbar]]\eea
with the induced Lie algebra structure such that the Lie bracket is defined by
\bea \lsb f,g\rsb\coloneqq \frac{1}{\hbar} \lsb f,g\rsb_\star=\frac{1}{\hbar}\lb f\star g- g\star f\rb.\eea

Let $\gsp_{2n}$ be the symplectic group of linear transformations preserving the Poisson bivector $\Pi$. It acts on Weyl algebras by inner automorphisms. We can identify the Lie algebra $\sp_{2n}$ of $\gsp_{2n}$ with the quadratic polynomial in $\bR[p_i,q^i]$, and $\sp_{2n}$ is a Lie subalgebra of $\cW_{2n}^+$.
The action $\gsp_{2n}\curvearrowright \bR^{2n}$ induces $\gsp_{2n}\curvearrowright \cW_{2n}^+$. Hence, $\lb \cW_{2n}^+,\gsp_{2n}\rb$ and $\lb \cW_{2n},\gsp_{2n}\rb$ are Harish-Chandra pairs.

Let $(X,\omega)$ be a symplectic manifold, and $F_{\gsp}(X)$ be the symplectic frame bundle. We have the Weyl bundles
\bea \cW^+_X=F_{\gsp}(X)\times_{\gsp_{2n}} \cW_{2n}^+, \quad 
\cW_X=F_{\gsp}(X) \times_{\gsp_{2n}} \cW_{2n}.\eea
Consider the Harish-Chandra pair
\bea (\ols{\fg},K)=(\fg/Z(\fg),\gsp_{2n}),\eea
where $\fg=\cW_{2n}^+$, and $Z(\fg)=\bR[[\hbar]]$ is the center of $\fg$, $Z(\fg)\cap \sp_{2n}=0$.
Fedosov constructed a flat $(\ols{\fg},K)$-bundle $F_{\gsp}(X)\to X$
and $H^0(X; \cW_X^+)$ gives a \emph{deformation quantization}. 
Choose the trivial $(\ols{\fg},K)$-module $\bR((\hbar))$. Then 
\bea \on{desc}: C^\blt_{Lie} \lb \ols{\cW^+_{2n}},\sp_{2n}; \bR((\hbar))\rb \to \Omega^\blt_X((\hbar)).\eea 
This is the \textbf{Gelfand-Fuks map}.
Here $C^\blt_{Lie} \lb \ols{\cW^+_{2n}},\sp_{2n}; \bR((\hbar))\rb \simeq C^\blt_{Lie}\lb \cW^+_{2n},\sp_{2n}\oplus Z(\cW^+_{2n}); \bR((\hbar))\rb$.

\subsection{Characteristic classes}
Let us review the Chern-Weil construction of \emph{characteristic classes} in Lie algebra cohomology.
They will descent to the usual characteristic forms via the Gelfand-Fuks map.

Let $\fg$ be a Lie algebra, and $\fh\subset \fg$ be its Lie subalgebra. Let the projection map
\bea \on{pr}: \fg\to\fh\eea
be the $\fh$-equivariant splitting of the embedding $\fh\subset \fg$. In general $\on{pr}$ is not a Lie algebra homomorphism from $\fg$ to $\fh$. The \emph{failure} of $\on{pr}$ being a Lie algebra homomorphism gives $R\in \on{Hom}\lb \asym^2\fg, \fh\rb$ by
\bea R(\alpha,\beta)=\lsb \on{pr}(\alpha),\on{pr}(\beta)\rsb_\fh-\on{pr}\lsb \alpha,\beta\rsb_\fg, \quad \alpha,\beta\in\fg.\eea
The $\fh$-equivariance of $\on{pr}$ implies that $R\in\on{Hom}_\fh\lb \asym^2(\fg/\fh),\fh\rb$. $R$ is called the \textbf{curvature form}.
Let $\sym^m(\fh^\vee)^\fh$ be $\fh$-invariant polynomials on $\fh$ of homogeneous degree $\on{deg}=m$. Given $P\in \sym^m(\fh^\vee)^\fh$, 
we can
associate a cochain $P(R)\in C^{2m}_{Lie}(\fg, \fh; \bR)$ by the composition
\bea P(R): \asym^{2m}\fg \xrightarrow{\asym^m R} \sym^m(\fh) \xrightarrow{P}\bR.\eea
It can be checked that $\p_{Lie} P(R)=0$, defining a cohomology class $[P(R)]$ in $H^{2m}(\fg,\fh;\bR)$ which does
not depend on the choice of $\on{pr}$. Therefore we have the analogue of Chern-Weil characteristic map
\bea \chi:\sym^\blt(\fh^\vee)^\fh \to H^\blt(\fg,\fh;\bR), \quad P\mapsto \chi(P) \coloneqq \lsb P(R)\rsb.\eea

Now we apply the above construction to the case where
\bea \fg= \cW^+_{2n}, \quad \fh=\sp_{2n}\oplus Z(\fg).\eea
Any element $f$ in $\fg= \cW^+_{2n}$ can be uniquely written as a polynomial
$f=f(y^i,\hbar)$, with coordinates $(y^1,\cdots,y^n,y^{n+1},\cdots, y^{2n})=(p_1,\cdots,p_n, q^1, \cdots, q^n)$. 
Define the $\fh$-equivariant projections
\bea \on{pr}_1(f) &=\left. \hf \sum_{i,j}\p_i \p_j f\right|_{y=\hbar=0} y^i y^j \in\sp_{2n},\\
\on{pr}_3(f) &=\left. f\right|_{y=0} \in Z(\fg).\eea
We obtain the corresponding curvature
\bea
R_1 &\coloneqq \lsb \on{pr}_1(-), \on{pr}_1(-)\rsb-\on{pr}_1\lsb-,-\rsb\ \in\on{Hom}(\asym^2 \fg,\sp_{2n}),\\
R_3 &\coloneqq -\on{pr}_3\lsb -,-\rsb\ \in\on{Hom}(\asym^2 \fg,\bR[[\hbar]]).
\eea

\begin{rmk}
A more general case can be considered when we incorporate vector bundles, where $\fg= \cW^+_{2n}+\hbar\lb \mathfrak{gl}\lb \cW^+_{2n}\rb\rb$, $\fh=\sp_{2n}\oplus \hbar\mathfrak{gl}\oplus Z(\fg)$. There the extra projection $\on{pr}_2$ and its corresponding curvature $R_2$ are defined as elements in $\hbar\mathfrak{gl}$ and $\on{Hom}\lb\asym^2, \mathfrak{gl}\rb$, respectively.
It is worthwhile to point out that all the $\on{Hom}$’s here are only $\bR$-linear map, but not
$\bR[[\hbar]]$-linear, although $\fg$ is a $\bR[[\hbar]]$-module.
\end{rmk}

We now define the \textbf{$\widehat{A}$-genus}
\bea\widehat{A}(\sp_{2n})\coloneqq \lsb \on{det}\lb \frac{R_1/2}{\sinh{(R_1/2)}}\rb^{\hf}\rsb \in H^\blt(\fg,\fh;\bR).\eea

\begin{prop}
Under the descent map $\on{desc}: H^\blt(\fg,\fh;\bR((\hbar)))\to H^\blt(X)((\hbar))$ via the Fedosov connection, we have
\bea \on{desc}\lb \widehat{A}(\sp_{2n})\rb &= \widehat{A}(X),\\
\on{desc}\lb R_3\rb &= \omega_\hbar- \hbar\omega.
\eea
\end{prop}

\subsection{Universal trace map}
Recall that using $\Omega^\blt_{S^1}\to \bR^{2n}$, we have obtained
\bea\Tr= \int_{BV}\circ \lan-\ran_{free}: CC^{per}_{-\blt}(\cW_{2n})\to \bK\coloneqq\bR((\hbar))[u,u^{-1}].\eea
Let us write 
\bea \Tr\in \on{Hom}_{\bK}\lb CC^{per}_{-\blt}(\cW_{2n}),\bK\rb.\eea
This is a $\lb \ols{\cW^+_{2n}},\gsp_{2n}\rb$-module. Via the flat $\lb \ols{\cW^+_{2n}},\gsp_{2n}\rb$-bundle $F_{\gsp}(X)\to X$, we obtain the associated bundle
\bea E^{per}\coloneqq F_{\gsp}(X) \times_{\gsp_{2n}}
\on{Hom}_{\bK}\lb CC^{per}_{-\blt}(\cW_{2n}),\bK\rb\eea
with induced flat connection $\nabla^\gamma$.

Recall the Weyl bundle $\cW(X)=F_{\gsp}(X) \times_{\gsp_{2n}} \cW_{2n}$ with flat connection $\nabla^\gamma$.
We would like to glue $\Tr$ on $X$. Let us denote $\delta$ for the differential on $\on{Hom}_{\bK}\lb CC^{per}_{-\blt}(\cW_{2n}),\bK\rb$ induced from $b+uB$. So \bea \delta \Tr=\Tr \lb (b+uB)(-)\rb=0.\eea
We can view $\Tr$ as defining an element in
\bea C^0_{Lie}\lb \fg,\fh; \on{Hom}_{\bK}\lb CC^{per}_{-\blt}(\cW_{2n}),\bK\rb\rb,\eea
where we take 
\bea\fg= \cW^+_{2n}/Z(\cW^+_{2n}), \quad \fh=\sp_{2n}.\eea
However, $\Tr$ is NOT $\fg$-invariant, i.e. $\p_{Lie}\Tr\neq 0$. In other words, $\Tr$ is NOT a map of $\lb\fg,\gsp_{2n}\rb$-module. So $\Tr$ can not be glued directly. 

It is observed that $\p_{Lie}\Tr=\delta(-)$. 
It turns out that we have a canonical way to lift $\Tr$ to
\bea \widehat{\Tr}\in C^\blt_{Lie}\lb \fg,\fh; \on{Hom}_{\bK}\lb CC^{per}_{-\blt}(\cW_{2n}),\bK\rb\rb\eea
such that
\bea\widehat{\Tr}=\Tr+ \text{terms in } C^{>0}_{Lie}\lb \fg,\fh; \on{Hom}_{\bK}\lb CC^{per}_{-\blt}(\cW_{2n}),\bK\rb\rb\eea
and satisfying the coupled cocycle condition
\bea \lb\p_{Lie}+\delta\rb \widehat{\Tr}=0.\eea
$\widehat{\Tr}$ is called the \textbf{universal trace map}. Let us insert $1\in \cW_{2n}$, then $\widehat{\Tr}(1)$ is $\p_{Lie}$-closed, which defines the \textbf{universal index}, $\lsb \widehat{\Tr}(1)\rsb\in H^\blt_{Lie}(\fg,\fh;\bK)$.

\begin{thm}[Universal algebraic index theorem]
\bea\lsb \widehat{\Tr}(1)\rsb=u^ne^{-R_3/(u\hbar)}\widehat{A}(\sp_{2n})_u,\eea
where for $A=\sum_{p \text{ even}}A_p$, $A_P\in H^p(\fg,\fh;\bK)$, \bea A_u=\sum_p u^{-p/2}A_p.\eea
\end{thm}
This theorem is developed in the works of Feigin-Tsygan \cite{feigin1989riemann}, Feigin-Felder-Shoikhet \cite{feigin2005hochschild}, Bressler-Nest-Tsygan \cite{bressler2002riemann}, and many others.

\begin{rmk}
This can be naturally generalized to the bundle case. See \cite{Gui:2019ldd}.
\end{rmk}

Now we apply the Gelfand-Fuks (descent) map on $\widehat{\Tr}$, such that
\bea\begin{tikzcd}
    C^\blt_{Lie}\lb \fg,\fh; \on{Hom}_{\bK}\lb CC^{per}_{-\blt}(\cW_{2n}),\bK\rb\rb \ar[d, "\on{desc}"]\\
    \Omega^\blt\lb X, \on{Hom}_{\bK}\lb CC^{per}_{-\blt}(\cW(X)),\bK\rb\rb
\end{tikzcd}\eea
Let $\cW_D(X)$ be the space of flat sections of $\cW(X)$
that gives a \emph{deformation quantization}. Then 
\bea
\on{desc}(\widehat{\Tr}): CC^{per}_{-\blt}(\cW_D(X))\to \Omega^\blt(X)((\hbar))[u,u^{-1}], \quad b+uB\mapsto d_X.
\eea
In particular, it defines a \emph{trace map} in deformation quantization by
\bea
f\in \cW_D(X)\mapsto \int_X \on{desc}(\widehat{\Tr})(f) \in \bR((\hbar)).
\eea
We can show that $\int_X \on{desc}(\widehat{\Tr})(f)$ does not involve $u$. By the \emph{universal algebraic index theorem}, we have
\bea \int_X \on{desc}(\widehat{\Tr})(1)=\int_X e^{-\omega_\hbar/\hbar}\widehat{A}(X).\eea
This gives the \textbf{algebraic index theorem}.

\paragraph{Construction of universal trace map $\widehat{\Tr}$.}
We have the following relations.
\bea\begin{tikzcd}
    \boxed{\text{background symmetry}} \ar[rr, leftrightarrow] \ar[dr,leftrightarrow] & & \boxed{\text{connection form}} \ar[dl,leftrightarrow]\\
    & \boxed{\text{interaction}} &
\end{tikzcd}\eea

Let $\Theta: \fg\to \cW^+_{2n}/ Z(\cW^+_{2n})=\fg$
be the canonical identity map.
For each $f\in \cW^+_{2n}/ Z(\cW^+_{2n})$, we have defined the local functional on $\cE=\Omega^\blt(S^1)\otimes \bR^{2n}$ by
\bea I_f(\varphi)=\int_{S^1} f(\varphi), \quad \varphi\in\cE.\eea
Then $\Theta$ gives a map
\bea I_\Theta: \fg\to \cO_{loc}(\cE), \quad f\mapsto I_{\Theta(f)}.\eea
We can view this map as 
\bea I_\Theta\in C^1(\fg,\cO_{loc}(\cE))=\fg^\vee\otimes \cO_{loc}(\cE).\eea
Now we can construct 
$\widehat{\Tr}\in C^\blt_{Lie}\lb \fg,\fh; \on{Hom}_{\bK}\lb CC^{per}_{-\blt}(\cW_{2n}),\bK\rb\rb$
by
\bea
\widehat{\Tr}\lb f_0\otimes f_1\otimes \cdots\otimes f_m\rb &\coloneqq
\int_{BV} \exp{\lb\hbar P^\infty_0\rb}
\lb \cO_{f_0,f_1,\cdots,f_m} 
e^{\frac{1}{\hbar}I_\Theta}\rb\ \in C^\blt(\fg,\fh;\bK), \quad f_i\in \cW_{2n}\\
&``=\int_{BV}\int_{\Im d^\ast\subset\cE} e^{-\frac{1}{2\hbar} \int_{S^1} \lan\varphi,d\varphi\ran+\frac{1}{\hbar}I_\Theta} \cO_{f_0,f_1,\cdots,f_m}\,''.
\eea

\subsection{Computation of index}
The Weyl algebra $\cW_{2n}$ can be viewed as a family of associative algebras parameterized by $\hbar$.
This leads to the \textbf{Getzler-Gauss-Manin connection} $\nabla_{\hbar \p_\hbar}\curvearrowright CC^{per}_{-\blt}(\cW_{2n})$.
The calculation of index consists of the following steps:
\bi[(1)]
\item \emph{Feynman diagram computation} implies
\bea \widehat{\Tr}(1)=u^n e^{-R_3/(u\hbar)}\lb\underbrace{\widehat{A}(\sp_{2n})_u}_{1-\text{loop computation}} +\cO(\hbar)\rb.\eea
\item Computation of \emph{Getzler-Gauss-Manin connection} shows 
$\nabla_{\hbar \p_\hbar} \lb e^{R_3/(u\hbar)}\widehat{
\Tr}(1)\rb$ is $\p_{Lie}$-exact.
\item Combining (1) and (2), we find 
\bea \lsb \widehat{
\Tr}(1)\rsb= \lsb u^n e^{-R_3/(u\hbar)} \widehat{A}(\sp_{2n})_u\rsb\ \in H^\blt(\fg,\fh;\bK).\eea
\ei

