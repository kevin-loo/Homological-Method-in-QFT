\section{Effective BV quantization}\label{sec:bv}
In the \nameref{sec:hla}, we discussed that
\bea \text{homotopy Lie algebra } (L_\infty-\text{algebra})\ \fg \LRA
\text{vector field } \delta \text{ on } \fg\lsb 1\rsb \text{ such that } \delta^2=0.\eea
We can write $\delta= \delta_1+\delta_2+\cdots$ where 
\bea \delta_k: \fg^\vee\lsb-1\rsb \to \sym^k\lb \fg^\vee\lsb-1\rsb\rb.\eea
For Lie algebra, $\delta=\delta_2$; for DGLA, $\delta=\delta_1+\delta_2$. 

\begin{conv}
Given $A,B\in \on{Hom}(V,V)$, we write
the \textbf{commutator} 
\bea \lsb A,B\rsb\coloneqq AB-(-1)^{|A||B|}BA,\eea
where $|A|$ is the degree of $A$. In particular, $\lsb A,B\rsb=AB+BA$ if $A,B$ are odd operators.
\end{conv}

\subsection*{BV master equation}
\begin{defn}
A \textbf{differential graded Batalin-Vilkovisky (DGBV) algebra} is a triple $\lb \cA,Q,\Delta\rb$ where
\bi[(1)]
\item $\cA$ is a graded commutative associative algebra,
\item $Q:\cA\to \cA$ is a derivation of $\on{deg} Q=1$ with $Q^2=0$,
\item $\Delta: \cA \to \cA$ is the BV operator, which is a second-order operator of $\on{deg}=1$ with $\Delta^2=0$,
\item $\lsb Q,\Delta\rsb=Q\Delta+\Delta Q=0.$
\ei
\end{defn}
Here $\Delta$ being \emph{second-order} means the following. Define the \textbf{BV bracket}
\bea \lcb a,b\rcb\coloneqq \Delta(ab)-(\Delta a)b-(-1)^{|a|} a\Delta b\quad \forall a,b\in \cA,\eea
which indicates the failure of $\Delta$ being a derivation. Then $\lcb -,-\rcb: \cA\otimes \cA\to \cA$ is a $\on{deg}=1$ operator satisfying
\bi[(1)]
\item $\lcb a,b\rcb= (-1)^{|a||b|}\lcb b,a\rcb$,
\item $\lcb a,bc\rcb=\lcb a,b\rcb c+(-1)^{\lb|a|+1\rb |b|} b\lcb a,c\rcb$,
\item $\Delta \lcb a,b\rcb=-\lcb \Delta a,b\rcb-(-1)^{|a|} \lcb a, \Delta b\rcb$.
\ei

\begin{eg}
Let $X$ be a smooth manifold and $\Omega$ be a volume form on $X$. Then we have the 
$\lb \on{PV}^\blt (X),\Delta_\Omega\rb$ complex, where $\Delta_\Omega: \on{PV}^{k}\to \on{PV}^{k-1}$ is the BV operator (or the divergence operator with respect to $\Omega$). Define $\lcb \alpha,\beta\rcb\coloneqq \Delta_\Omega\lb \alpha\beta\rb- \lb\Delta_\Omega \alpha\rb \beta-(-1)^{|\alpha|}\alpha\Delta_\Omega\beta$. Then $\lcb -,-\rcb$ is the \textbf{Schouten-Nijenhuis bracket} (up to a sign).
\end{eg}

\begin{fact}
$\lcb -,-\rcb$ does not depend on the choice of $\Omega$.
\end{fact}

\begin{eg}
Let $X$ be a Calabi-Yau manifold and $\Omega$ be a holomorphic volume form. The polyvector field is
\bea \on{PV}^{k,l}(X)=\Omega^{0,l} \lb X,\asym^k T_X^{1,0}\rb,\eea
the Dolbeault differential operator is 
\bea \pb: \on{PV}^{k,l}\to \on{PV}^{k,l+1},\eea
the BV operator (i.e. the divergence operator with respect to $\Omega$) is
\bea \Delta: \on{PV}^{k,l}\to \on{PV}^{k-1,l}.\eea
Then $\lb \on{PV}^{-\blt,\blt}(X), \pb, \Delta\rb$ is a DGBV.
\end{eg}

\begin{rmk}
The existence of such structure implies that the local moduli of complex structure of the Calabi-Yau manifold $X$ is smooth. This follows from the Bogomolov-Tian-Todorov (BTT) lemma.
\end{rmk}

\begin{defn}
Let $\lb \cA,Q,\Delta\rb$ be a DGBV. Suppose an element $I_0\in \cA_0$, where $\on{deg}(I_0)=0$. $I_0$ is said to satisfy \textbf{classical master equation (CME)} if  
\bea QI_0+\hf \lcb I_0,I_0\rcb=0.\eea
This implies $\lb Q+\lcb I_0,-\rcb\rb^2=0$ on $\cA$.
In good situation, we can write
\bea  Q+\lcb I_0,-\rcb=\lcb S_0,-\rcb,\eea
where $S_0$ is the classical action and $I_0$ is its interaction part. 
Then the CME means $\lcb S_0, S_0\rcb=0$.
\end{defn}

The upshot here is that any classical action along with gauge symmetry leads to a solution of CME.

\begin{defn}
$I\in \cA_0\llb \hbar\rrb$ is said to satisfy the \textbf{quantum master equation (QME)} if 
\bea QI+ \hbar \Delta I+\hf \lcb I,I\rcb =0.\eea
We can write $I=I_0+\hbar I_1+\hbar^2 I_2+\cdots$. When $\hbar\to 0$, the above equation reduces to 
\bea QI_0+\hf \lcb I_0,I_0\rcb=0. \eea
In good situation (in QFT), we write
\bea  Q+\lcb I,-\rcb=\lcb S,-\rcb,\eea
where $S$ is the quantum action.
Then the QME means $\hbar \Delta S+\hf \lcb S, S\rcb=0$.
\end{defn}
In QFT, the quantization process asks the classical action $S_0$ solving the CME to be quantized to the quantum action $S=S_0+\hbar S_1+\hbar^2 S_2+\cdots$ solving the QME.

\begin{eg}
QME $\LRA (Q+\hbar \Delta) e^{I/\hbar}=0$
(or in good situation $\Delta e^{S/\hbar}=0$).
\end{eg}

\paragraph{Toy model: (-1)-shifted symplectic geometry.}
Let $(V, Q,\omega)$ be a finite dimensional differential graded symplectic space.
\begin{itemize}
    \item $Q: V\to V$ is the differential operator with $Q^2=0$, 
    \item $\omega: \asym^2 V\to \bR$ or $\bC$ is a non-degenerate pairing of $\on{deg}=-1$, or explicitly $\omega(a,b)=0$ unless $|a|+|b|=1$,
    \item $Q(\omega)=0$, or explicitly  $\omega(Q(a),b)+(-1)^a\omega(a, Q(b))=0$.
\end{itemize}
The \emph{non-degeneracy} of $\omega$ implies an isomorphism
\bea \omega: V^\vee \xrightarrow{\ \sim\ } V\lsb 1\rsb. \eea
This allows us to identify
\bea \asym^2 V^\vee \xleftrightarrow{\ \sim\ } \asym^2 \lb V[1]\rb \simeq \sym^2(V)[2], \qquad \omega\in \asym^2 V^\vee \xleftrightarrow{\ \sim\ } K[2],\eea
where $K=\omega^{-1}\in \sym^2(V)$ is the \textbf{Poisson kernel} with $\on{deg} K=1$ and $Q(K)=0$.
We obtain a DGBV $\lb \cA,Q,\Delta_K\rb$ as follows. 
\begin{itemize}
    \item $\cA=\sO(V)=\widehat{\sym}(V^\vee)$ is a space of formal functions on $V$,
    \item $Q:\cA\to\cA$ is the differential induced from $Q:V\to V$,
    \item $\Delta_K: \sym^m (V^\vee)\to \sym^{m-2} (V^\vee)$ is the contraction with the Poisson kernel $K\in \sym^2(V)$. Explicitly, for $\alpha_i\in V^\vee$, 
    \bea \Delta_K(\alpha_1\otimes\cdots \otimes\alpha_m)=\sum_{i<j}\pm \lan K,\alpha_i\otimes \alpha_j\ran \alpha_1\otimes \cdots \otimes \widehat{\alpha_i} \otimes \cdots \otimes \widehat{\alpha_j} \otimes \cdots \otimes \alpha_m.\eea
    Here $\pm$ is the \emph{Koszul sign} by permuting graded objects.
\end{itemize}

\begin{prop}
$(\sO(V),Q,\Delta_K)$ is a DGBV.
\end{prop}
\begin{proof}
Exercise.
\end{proof}

Let $S_0\in \sO(V)$ solve the CME:
\bea \lcb S_0,S_0\rcb=0 \quad (\on{deg} S_0=0).
\eea
Then $\delta=\lcb S_0,-\rcb$ defines a vector field on $V$ with $\on{deg} \delta=1, \delta^2=0$. Hence $(\fg=V[-1], \delta)$ is an $L_\infty$-algebra.

\subsection*{Field theory via BV formalism}
A classical field theory can be usually organized into an $\infty$-dimensional $(-1)$-symplectic geometry $(\cE,Q,\omega)$ where
\bi[(1)]
\item $\cE=\Gamma\lb X,E^\blt\rb$ is the space of fields (or space of sections), where $E^\blt$ is the graded vector bundle,
\item $(\cE,Q)$ is an elliptic complex, in which the differential $Q$ is naturally defined as follows.
\bea \cdots \longrightarrow \cE^{-1}\xrightarrow{\ Q\ }\cE^0 \xrightarrow{\ Q\ }\cE^1 \longrightarrow\cdots.\eea
$Q$ is an elliptic operator. For example, $Q$ can be the de Rham differential $d$ or the Dolbeault differential $\pb$.
\item $\omega$ is the local $(-1)$-symplectic pairing, where
\bea \omega(\alpha,\beta)= \int_X \lan \alpha,\beta\ran \quad \forall \alpha,\beta\in\cE\eea
and is compatible with $Q$:
\bea \omega(Q(\alpha),\beta)+(-1)^\alpha \omega(\alpha,Q(\beta))=0.\eea
\ei

To describe the quantization, we perform the \emph{toy model}, in which we define
\begin{itemize}
    \item $\cE^\vee \coloneqq\on{Hom}_X(\cE,\bR)$ is a distribution on $X$,
    \item $(\cE^\vee)^{\otimes n}=\on{Hom}_{X^n}(\cE^{\otimes n},\bR)$ is a distribution on $X^n$,
\end{itemize}
so $\sym^n(\cE^\vee)$ is defined.
Let us form 
\bea \sO(\cE)\coloneqq \prod_{n\geq 0}\sym^n(\cE^\vee).\eea
Let $\sO_{loc}(\cE)\subset \sO(\cE)$ be a \textbf{space of local functionals}, which contains the set 
\bea\lcb \left. \int_X \cL\ \right|\ \cL \text{ is the Lagrangian density} \rcb.\eea
The Poisson kernel is $K=\omega^{-1}=$``$\lb \int_X \lan \alpha,\beta\ran\rb^{-1}$'', which is a $\delta$-function distribution as in $f(x)=\int dy \delta_{x,y} f(y) $. $K$ is a \textbf{distributional section} of $\sym^2(\cE)$.

However, there is one problem. The action of $\Delta_K$ on $\sO(\cE)$ ($\Delta_K \curvearrowright\sO(\cE)$) is ill-defined since it is a pairing of two distributions, each of which is in infinite-dimensional space, leading to $\infty$. This is the famous \textbf{ultraviolet problem}.

\begin{eg}
In classical sense, the corresponding BV bracket $\lcb -,-\rcb$ is well-defined on local functionals $\sO_{loc}(\cE)$:
\bea \lcb -,-\rcb: \sO_{loc}(\cE)\times \sO_{loc}(\cE)\to \sO_{loc}(\cE).\eea
\bea
    \begin{fmffile}{fdd}
    \begin{tabular}{c}
        \begin{fmfgraph*}(150,50)
                \fmfleft{i1,i2,i3}
                \fmfright{o1,o2,o3}
                \fmf{plain,tension=4}{i1,v1}
                \fmf{plain,tension=4}{i2,v1}
                \fmf{plain,tension=4}{i3,v1}
                \fmf{plain,tension=4}{v2,o1}
                \fmf{plain,tension=4}{v2,o2}
                \fmf{plain,tension=4}{v2,o3}
                \fmf{plain,label=$\delta_{x,,y}$,label.side=left,tension=4}{v1,v2}
                \fmfv{label=$\int_X\cL_1$,label.angle=-60,decor.shape=circle,decor.filled=full,decor.size=2thick}{v1}
                \fmfv{label=$\int_X\cL_2$,label.angle=-120,decor.shape=circle,decor.filled=full,decor.size=2thick}{v2}
        \end{fmfgraph*}
        \end{tabular}
    \end{fmffile}
    ~~~ =\ \int_X \lb \cdots\rb.
\eea
Hence, CME makes sense for local functionals. But for quantization, $I_0\to I=I_0+\hbar I_1+\cdots$. We have
\bea QI+\hbar \Delta I+\hf \lcb I,I\rcb=0.\eea
The term $\hbar \Delta I$ is problematic and can be solved by the renormalization technique.
\end{eg}

\begin{eg}[Chern-Simons (CS) theory]
Let $X$ be a 3-manifold, $\fg$ be the Lie algebra, and $\Tr$ be the Killing pairing on $\fg$.
\begin{itemize}
    \item $\cE=\Omega^\blt\lb X,\fg[1]\rb$ where
        \begin{table}[!htpb]
            \centering
            \begin{tabular}{l|cccc}\toprule
            $\mathbf{\Omega^\blt}$ & $\Omega^0(X,\fg)$ & $\Omega^1(X,\fg)$ & $\Omega^2(X,\fg)$ & $\Omega^3(X,\fg)$\\ \hline
            \textbf{deg} & -1 & 0 & 1 & 2\\ \hline
            \textbf{name} & ghost & field (connection) & anti-field & anti-ghost\\ \bottomrule
            \end{tabular}\end{table}
        
    \item $\omega(\alpha,\beta)=\pm \int_X \Tr \lan \alpha,\beta\ran, \quad \on{deg}\omega=-1$,
    \item The CS action functional is 
    \bea\on{CS}[\cA]=\int \Tr\lb \hf \cA\wedge d\cA+\frac{1}{6}\cA\wedge [\cA,\cA]\rb,\eea 
    where the first term is the free part of the action and the second term is the interaction part, $I$. Here the master field is \bea\cA=c+A+A^\vee+c^\vee=\Omega^0+\Omega^1+\Omega^2+\Omega^3 \in \cE.\eea
    Hence,
    \bea\on{CS}[\cA]=\int \Tr\lb \hf A\wedge dA+\frac{1}{6}A\wedge [A,A]\rb+ \text{terms containing ghosts},\eea 
\end{itemize}

\begin{clm}
$\on{CS}[\cA]$ satisfies the CME: $\lcb \on{CS}, \on{CS}\rcb=0$.
\end{clm}
If we write $\on{CS}=\text{free}+I$ as above, then $\lcb \text{free},-\rcb=d$. Hence,
$\text{CME}\LRA dI+\hf \lcb I,I\rcb=0$.
One way to see this is that $\Omega^\blt(X,\fg)$ is a DGLA $\RA$ CME.
\end{eg}

\subsection*{Costello's homotopic renormalization}
We have seen that the action
\bea
\Delta_K \curvearrowright \sO(\cE)
\eea
is ill-defined naively, in which $K=\omega^{-1}$ is singular.

\paragraph{Toy model.} 
Let $(V,Q)$ be a finite dimensional complex and $K_0\in \sym^2(V)$ be the Poisson kernel with $\on{deg} K_0=1$ and $Q(K_0)=0$. Then $\Delta_0: \sO(V)\to \sO(V)$ is a BV operator by contracting with $K_0$. Let $P\in \sym^2(V)$, $\on{deg} P=0$. Consider the \emph{chain homotopy} for BV kernel:
\bea K_P=K_0+Q(P)=K_0+(Q\otimes 1+1\otimes Q)P,\eea
then $\on{deg}(K_P)=1$, $Q(K_P)=QK_0+Q^2(P)=0$. We can consider a new BV operator
$\Delta_P$, which is a contraction with $K_P$.

\begin{prop}
$\lb Q+\hbar \Delta_P\rb e^{\hbar\p_P}=e^{\hbar\p_P}\lb Q+\hbar \Delta_0\rb$.
\end{prop}
Here $\p_P$ is the second order operator of contracting with $P\in \sym^2(V)$:
\bea \p_P: \sym^n(V^\vee)\to \sym^{n-2}(V^\vee).\eea
Then we have the commutative diagram
\bea
\begin{tikzcd}
\sO(V)\llb \hbar\rrb \ar[r, "e^{\hbar\p_P}"] \ar[d,"Q+\hbar \Delta_0"]
& \sO(V)\llb \hbar\rrb \ar[d,"Q+\hbar \Delta_P"] \\
\sO(V)\llb \hbar\rrb \ar[r, "e^{\hbar\p_P}"]
& \sO(V)\llb \hbar\rrb
\end{tikzcd}
\eea
which implies that
\bea \text{QME for } (\sO(V),Q,\Delta_0) \xrightarrow{e^{\hbar \p_P}}
\text{QME for } (\sO(V),Q,\Delta_P),\eea
where $e^{\hbar \p_P}$ is the homotopy RG flow. Hence,
\bea \lb Q+\hbar \Delta_0\rb e^{I/\hbar}=0 \LRA \lb Q+\hbar \Delta_P\rb e^{\Tilde{I}/\hbar}=0\eea
where $e^{\Tilde{I}/\hbar}=e^{\hbar \p_P}e^{I/\hbar}$.
As we have seen before, this can be expressed by the Feynman diagrams:
\bea \Tilde{I}=\sum_{\text{connected graph}}\lb 
    \begin{fmffile}{fddi}
    \begin{tabular}{c}
        \begin{fmfgraph*}(100,50)
                \fmfleft{i1,i2}
                \fmfright{o1,o2}
                \fmf{plain,tension=4}{i1,v1}
                \fmf{plain,tension=4}{i2,v1}
                \fmf{plain,tension=4}{v2,o1}
                \fmf{plain,tension=4}{v2,o2}
                \fmf{plain,left,label=$P$,label.side=left,tension=3}{v1,v2,v1}
                \fmfv{label=$I$,label.angle=170,decor.shape=circle,decor.filled=full,decor.size=2thick}{v1}
                \fmfv{label=$I$,label.angle=10,decor.shape=circle,decor.filled=full,decor.size=2thick}{v2}
        \end{fmfgraph*}
        \end{tabular}
    \end{fmffile}\rb.
\eea

\paragraph{Back to QFT.}
Consider the $\infty$-dimensional $(-1)$-symplectic geometry $(\cE=\Gamma(X,E^\blt),Q,\omega)$ with the $\delta$-function distribution $K_0=\omega^{-1}$ and $Q(K_0)=0$.

\noindent\emph{Costello's approach}: use the elliptic regularity:
\bea H^\blt(\text{distributions}, Q)=H^\blt(\text{smooth functions}, Q).\eea 
We can write $K_0=K_r+QP_r$, where $K_r$ is smooth and $P_r$ is singular, called the \emph{parametrix}. $\Delta_r:\sO(\cE)\to \sO(\cE)$ is the BV operator contracting with $K_r$, which is well-defined since $K_r$ is smooth. Hence $\lb \sO(\cE),Q,\Delta_r\rb$ is the ``effective'' DGBV.

Let $r'$ be another regularization such that
\bea K_0=K_{r'}+QP_{r'}.\eea
Hence we find a smooth kernel $P^{r'}_r$ from 
\bea K_{r'}-K_r=QP^{r'}_r.\eea
Let $\p_{P^{r'}_r}: \sO(\cE)\to \sO(\cE)$ be contracting with the smooth kernel $P^{r'}_r$. Hence, we have a homotopy RG flow:
\bea \lb\sO(\cE)\llb \hbar\rrb,Q+\hbar\Delta_r\rb \xrightarrow{\exp{\lb \hbar \p_{P^{r'}_r}\rb}}
\lb\sO(\cE)\llb \hbar\rrb,Q+\hbar\Delta_{r'}\rb.\eea

\begin{defn}[Costello]
An effective solution of perturbative BV quantization of $I_0$ (which solves CME) is given by a family $I[r]\in \sO(\cE)\llb \hbar\rrb$ (for each choice of regularization $P_r$) such that
\begin{itemize}
    \item $\lb Q+\hbar \Delta_r\rb e^{I[r]/\hbar}=0$ is the effective QME,
    \item $e^{I[r']/\hbar}=\exp{\lb \hbar \p_{P^{r'}_r}\rb} e^{I[r]/\hbar}$ is the homotopy RG flow, or equivalently,
    \bea I[r']=\sum_{\text{connected graph}}\lb 
    \begin{fmffile}{fddj}
    \begin{tabular}{c}
        \begin{fmfgraph*}(100,50)
                \fmfleft{i1,i2}
                \fmfright{o1,o2}
                \fmf{plain,tension=4}{i1,v1}
                \fmf{plain,tension=4}{i2,v1}
                \fmf{plain,tension=4}{v2,o1}
                \fmf{plain,tension=4}{v2,o2}
                \fmf{plain,left,label=$P$,label.side=left,tension=3}{v1,v2,v1}
                \fmfv{label=$I[r]\ $,label.angle=170,decor.shape=circle,decor.filled=full,decor.size=2thick}{v1}
                \fmfv{label=$I[r]$,label.angle=10,decor.shape=circle,decor.filled=full,decor.size=2thick}{v2}
        \end{fmfgraph*}
        \end{tabular}
    \end{fmffile}\rb,
\eea
\item $I[r]$ is asymptotic local when $r\to 0$ and $I_0=\lim_{r\to 0}\lim_{\hbar\to 0}I[r]$. 
\end{itemize}
\end{defn}

\noindent \textsc{References}:
\cite{costello2011renormalization} for Costello's homotopy renormalization theory, \cite{Li:2016gcb} for the style of the lecture.
